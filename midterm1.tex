% Created 2022-09-20 Tue 22:53
% Intended LaTeX compiler: pdflatex
\documentclass[11pt]{article}
\usepackage[utf8]{inputenc}
\usepackage[T1]{fontenc}
\usepackage{graphicx}
\usepackage{longtable}
\usepackage{wrapfig}
\usepackage{rotating}
\usepackage[normalem]{ulem}
\usepackage{amsmath}
\usepackage{amssymb}
\usepackage{capt-of}
\usepackage{hyperref}
\author{Theo Park}
\date{\today}
\title{CS250 Midterm 1 Review}
\hypersetup{
 pdfauthor={Theo Park},
 pdftitle={CS250 Midterm 1 Review},
 pdfkeywords={},
 pdfsubject={},
 pdfcreator={Emacs 28.1 (Org mode 9.5.2)}, 
 pdflang={English}}
\begin{document}

\maketitle
\setcounter{tocdepth}{2}
\tableofcontents


\section{Why Computer Architecture}
\label{sec:org86fa1b2}

\subsection{Definitions}
\label{sec:org26f6d12}

\begin{itemize}
\item \emph{Computer} is a machine that can be programmed to \textbf{carry out computation automatically}
\item \emph{Architecture} is a \textbf{conceiving, planning, and designing structures}
\begin{itemize}
\item CA has purpose only when given SW
\end{itemize}
\item \emph{Software} is a \textbf{description of a computation} expressed in a programming language, any data, and documentation
\begin{itemize}
\item Purpose 1: Definining an DS \& A
\item Purpose 2: Executing
\end{itemize}
\item \emph{Interpreter} \textbf{executes software}
\begin{itemize}
\item Directly executes instructions expressed in a PL
\item \textbf{Does NOT rely on "Turtles all the way down"} (interpreter for interpreter for interpreter\ldots{}) approach
\end{itemize}
\item \emph{Compiling} is the process of \textbf{traslating} programs written in one \textbf{HLL} (High-level language) into a \textbf{LLL} that \textbf{has a machine interpreter}
\end{itemize}

\subsection{C Compiling Process}
\label{sec:orge8e59c5}

// TODO Find a better way to put diagram in Org files
\begin{verbatim}
source_code -> preprocessor -> preprocessed source code -> compiler -> assembly code -> assembler -> relocatable object code -> linker (w/ object code in lib) -> binary object code
\end{verbatim}

\begin{itemize}
\item Preprocessed Source Code: Does not contain \textbf{comments, macros, includes}, etc
\item Assembly Code: \textbf{Machine specific}
\end{itemize}

\subsection{Mechanical Computers}
\label{sec:org5e31303}

\begin{itemize}
\item Antikythera Mechanism (200B.C): Count Olumpics days
\item Charles Babbage (1849)
\end{itemize}

\subsubsection{Disadvantages}
\label{sec:org08dc56e}

\begin{itemize}
\item Parts are small, require individual assembly
\item Part shape and size determine computational function
\item Parts cause waer and accuracy degrades over time
\item Algorithm are slow
\end{itemize}

\subsection{Vacuum Tube Computers}
\label{sec:orge174ab7}

\begin{itemize}
\item Colossus
\end{itemize}

\subsubsection{Disadvantages}
\label{sec:orgcf6a44e}

\begin{itemize}
\item About the same volume as mechanical computer
\item Uses a lot of electrical energy
\item Vacuum tubes burn out
\end{itemize}

\subsection{Transistor}
\label{sec:orgd218750}

\begin{itemize}
\item First one built at AT\&T Bell Labs
\item Used to use germanium crystal, now use silicon
\item Futures are graphene or single layer of carbon
\end{itemize}

\subsection{Two Architectures}
\label{sec:orgd2e215f}

\subsubsection{Harvard Architecture}
\label{sec:orgd7df494}

\textbf{Separate memories} for instructions and data

\subsubsection{Von Neumann Architecture}
\label{sec:orgc02bc5b}

\textbf{Single memory} for instruction and data

\section{Representation}
\label{sec:orgcb94292}

\subsection{Electrical Representation of Bits}
\label{sec:orgb2669b5}

\begin{itemize}
\item \textbf{V (max) voltage V - \(\Delta\)} is recognizes as 1
\item \textbf{0 to 0 + \(\delta\)} is recognizes as 0
\item \textbf{Rising edge} and \textbf{falling edge} are ignored
\end{itemize}

\subsection{Bit String}
\label{sec:org3427505}

\begin{itemize}
\item \textbf{Bus: Collection of k wires carrying k-bits}
\item \textbf{k-bits} on \textbf{k-wires}
\item k-bits can represent up to \textbf{\(2^k\) values}
\item \emph{Bit strings are only meaningful when it is paried with a representation}
\end{itemize}

\section{Regular Representations}
\label{sec:orgb91683d}

Unsigned and 2's complement integers are native data types for most modern circuits

\subsection{Unsigned integer, base 2, weighted positional}
\label{sec:org44b8e1f}

\textbf{Regular binary number} that we think of normally.

\texttt{001011} = \(0 \times 2^5 + 0 \times 2^4 + 1 \times 2^3 + 0 \times 2^2 + 1 \times 2^1 + 1 \times 2^0 = 11\)

\subsection{Sign Magnitude}
\label{sec:orge8535e3}

\textbf{UIB2WP but the MSB is the sign} (MSP = left most bit).

\texttt{101011} = \(-1(0 \times 2^4 + 1 \times 2^3 + 0 \times 2^2 + 1 \times 2^1 + 1 \times 2^0) = -11\)

\subsubsection{Characteristics of sign magnitude}
\label{sec:org5c7d03c}

\begin{itemize}
\item There are two zeros (0000 = +0, 1000 = -0)
\item Less number can be represented (duh)
\end{itemize}

\subsection{Two's Complement}
\label{sec:orgb51f8ed}

\textbf{MSB weight is negative}

\texttt{101011} = \(-(1 \times 2^4) + 1 \times 2^3 + 0 \times 2^2 + 1 \times 2^1 + 1 \times 2^0) = -5\)

\subsubsection{Characteristics of two's complement}
\label{sec:org0b650da}

\begin{itemize}
\item Only one bit string for zero
\item \textbf{Invert bit string and add 1 to get the negative}
\item \textbf{Uses the same circuit as unsigned integer add/subtraction}
\end{itemize}

\section{Casting/Sign Extension}
\label{sec:orgf71205e}

\begin{itemize}
\item Unsigned integer: \textbf{Add 0 in front}
\item 2's complement: \textbf{Add MSB in front}
\end{itemize}

\section{Overflow}
\label{sec:org76f1629}

\begin{itemize}
\item Adding two k-bit unsigned integer resulting in (k+1)-bit result
\item \(A +_k B = (A+B) \text{ mod } k\) prevents it
\end{itemize}

\section{Gray Code}
\label{sec:org5e68a09}

For sensors where bits need to be detected fast, "gray code" where only one bit changes per number is used.

\section{ASCII}
\label{sec:org69314f6}

\subsection{History}
\label{sec:orgac3f0a8}

\emph{Baudot Code} in 1870 used to represent \(2^5\) characters with 5 keys.

\subsubsection{Design of ASCII}
\label{sec:orgfc3d6bd}

\begin{itemize}
\item \textbf{Designed for machine, not human}
\item Alphabetic order = integer order of chracter codes
\item Upper and lower case only differ in \textbf{bit 7, the MSB}
\end{itemize}

\subsubsection{Unicode}
\label{sec:org38ed606}

\begin{itemize}
\item Up to 4 bytes per character
\item Currently 14.0, supports emoji
\end{itemize}

\section{Order of Bytes in Memory}
\label{sec:org2400e91}

\begin{itemize}
\item \emph{Big Endian}: \textbf{MSB comes first}
\texttt{0x5060} is stored as \texttt{0x5060}
\item \emph{Lil Endian}: \textbf{LSB comes first}
\texttt{0x5060} is stored as \texttt{0x6050}
\end{itemize}

\section{Floating Point Representation (IEEE 754)}
\label{sec:org98c6887}

\texttt{|S| Exponent | mantissa |}

\subsection{Exponent}
\label{sec:org4e253a9}

Exponent is a biased integer. The initial range is -127 < e < 127, sign is made implicit by E = e + Bias = e + 127.

\subsection{Mantissa}
\label{sec:orgf5e30b8}

Unless the number is 0, the MSB of the mantissa must be 1 -> No need to store! (\textbf{hidden bit}).
Instead, \textbf{one extra precision bit} is stored in the end.

\subsection{Runtime Anomalies}
\label{sec:org1339181}

\begin{enumerate}
\item \emph{E = 0, Mantissa = 0,}: \(\pm 0\), depending on the sign bit
\item \emph{E = 0, Mantissa \(\neq\) 0, Mantissa MSB = 0}: De-normalized number, gradual underflow
\item \emph{E = 255, Mantissa = 0}: \(\pm \infty\); in general, overflow is set to infinity to help people
\item \emph{E = 255, Mantissa \(\neq\) 0}: Not a Number
\end{enumerate}


\section{Memory}
\label{sec:org4022c45}

\subsection{Pointing Function}
\label{sec:org7ac8b26}

\begin{figure}[htbp]
\centering
\includegraphics[width=.9\linewidth]{./img/pointing_function.jpg}
\caption{\label{fig:org2ff79ef}Abstract Pointing Function Diagram}
\end{figure}

Memory chips are 2D, \(2^{k/2} \times 2^{k/2}\) grid creates \(2^k\) intersections -> Only \(2 * 2^{k/2}\) wires needed! Optimized design can reduce number of wires by \(\sqrt{2^k} / 2\)

\subsection{Register}
\label{sec:orga83763c}

k-bit register has k latches to store \(2^k\) bit strings.

\section{Pointing}
\label{sec:org2e36d06}

\subsection{Decoder}
\label{sec:org9b1b19c}

A circuit with n wires input and 2\textsuperscript{n} wires output. Decoding output is selected or not\textsubscript{selected}

\begin{center}
\begin{tabular}{rrrrrr}
X & Y & D0 & D1 & D2 & D3\\
\hline
0 & 0 & 1 & 0 & 0 & 0\\
1 & 0 & 0 & 1 & 0 & 0\\
0 & 1 & 0 & 0 & 1 & 0\\
1 & 1 & 0 & 0 & 0 & 1\\
\end{tabular}
\end{center}

2-to-4 decoder truth table. \textbf{n inputs and 2\textsuperscript{n} outputs}.

\subsection{Selecting bus}
\label{sec:org703d9d1}

\begin{itemize}
\item \emph{Bus}: Group of n wires, carry n bit
\item \emph{Multiplexer (mux)}: Selects \textbf{from 2\textsuperscript{n} k-bit input buses, outputs to 1 k-bit output bus}
\item \emph{Demultiplexer (Demux)}: Reverse of mux
\end{itemize}

\section{Processor}
\label{sec:orgcd07648}

\subsection{Revisit of Architectures}
\label{sec:org4ff58a5}

\begin{itemize}
\item Harvard: Optimized design and simultaneous access for data and instructions; storage inefficiency
\item Von Neumann: Less memory needed; security issue
\end{itemize}

\subsection{Processor is}
\label{sec:orgbbd534d}

\begin{itemize}
\item Not CPU
\item Includes co-processor and microcontroller
\item Building specific one purpose is expensive -> General purpose processor
\end{itemize}

\section{General One-Step Processor Circuit}
\label{sec:orgf908c41}

\begin{figure}[htbp]
\centering
\includegraphics[width=.9\linewidth]{./img/general_one_step_processor_circuit.jpg}
\caption{\label{fig:orgd9ca102}General One-Step Processor Circuit}
\end{figure}

Key points:
\begin{itemize}
\item Input goes from register collection to MUX1 and MUX2, they choose two k-bit strings
\item ALU \textbf{continously} to the calculation and outputs to the MUX3
\item MUX3 chooses one of the result and outputs
\item DeMUX put the bit string back to \textbf{correct} location (MUX3 will mess it up if it puts it back to the collection w/o DeMUX)
\end{itemize}

\subsection{Fetch-Execute Cycle}
\label{sec:org0c714f7}

\begin{verbatim}
while (power is on) {
  fetch;
  execute;
}
\end{verbatim}

Wow fancy algorithm.

\subsection{Clock rate and Instruction rate}
\label{sec:orgd6b9939}

\begin{itemize}
\item Clock rate: Worst circuit propagation delay
\item Registers and mux/demux delay exists
\item ALU propagation delay varies a lot since it's where all the calculations happen
\end{itemize}

\begin{align*}
\text{CPU Time} = \frac{Instructions}{Program} \times \frac{Clock Cycles}{Instruction} \times \frac{Seconds}{Clock cycle}
\end{align*}

\begin{itemize}
\item Instruction per program: Software runtime dependent. 251 flashback
\item Clock cycles per instruction: Compiler and circuit design dependent
\item Seconds per clock cycle: Worst case propagation delay, the clock rate of the CPU
\end{itemize}

\subsection{Start and Stopping Hardware}
\label{sec:orgaf38226}

Hardware is designed to run 24/7. For your computer, there is an idle loop to run to continue fetch-execute cycle.

Bootstrap: \textbf{power-on reset} of latches to put the computer in known state to start fetch-execute cycle, 

\section{Instruction Encoding}
\label{sec:org57c6d89}

\subsection{Instruction Set Architecture}
\label{sec:orgc5188a6}

Set of operations chosen by a careful consideration to make processors the more expensive.

\begin{verbatim}
| opcode | operand 1 | operand 2 | ... | Result 1 | Result 2 | ...
\end{verbatim}

\begin{itemize}
\item \textbf{Opcode}: Selects ALU result to use, contains number of operands
\item \textbf{Operands}: Provide input to ALU
\item \textbf{Results}: Pointers to storage locations
\end{itemize}

\subsection{Instruction Size}
\label{sec:org9e623a1}

\subsubsection{Variable Length}
\label{sec:orgaf7025a}

\begin{itemize}
\item Marketing people loves it
\item Compiler people hates it
\item Computer slow
\item Complex
\end{itemize}

\subsubsection{Fixed Length}
\label{sec:orgc09d553}

\begin{itemize}
\item Marketing people hates it
\item Compiler people loves it
\item Ez
\item Instructions may not utilize all the bits
\end{itemize}

\subsubsection{What drives the length?}
\label{sec:org8e6779a}

\begin{itemize}
\item Not opcode, \(2^k \times 2 = 2^{k+1}\) only 1 bit is needed to double the operations
\item Operands, results fields are the bad guys
\end{itemize}

\subsubsection{Fix Pointer Size}
\label{sec:orge8639a0}

\begin{itemize}
\item Make pointer size a constant
\item ALU circuits access memory, fetch operands and store result in fixed number of registers
\item Main memory grow, pointer unchanged
\end{itemize}

\section{x86}
\label{sec:orgb5b1ca9}

Intel rich.
\end{document}