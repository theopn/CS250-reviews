% Created 2022-09-17 Sat 16:29
% Intended LaTeX compiler: pdflatex
\documentclass[11pt]{article}
\usepackage[utf8]{inputenc}
\usepackage[T1]{fontenc}
\usepackage{graphicx}
\usepackage{longtable}
\usepackage{wrapfig}
\usepackage{rotating}
\usepackage[normalem]{ulem}
\usepackage{amsmath}
\usepackage{amssymb}
\usepackage{capt-of}
\usepackage{hyperref}
\author{Theo Park}
\date{\today}
\title{CS250 Midterm 1 Review}
\hypersetup{
 pdfauthor={Theo Park},
 pdftitle={CS250 Midterm 1 Review},
 pdfkeywords={},
 pdfsubject={},
 pdfcreator={Emacs 28.1 (Org mode 9.5.2)}, 
 pdflang={English}}
\begin{document}

\maketitle
\setcounter{tocdepth}{2}
\tableofcontents


\section{Why Computer Architecture}
\label{sec:org82c2011}

\subsection{Definitions}
\label{sec:org31cdfd2}

\begin{itemize}
\item \emph{Computer} is a machine that can be programmed to \textbf{carry out computation automatically}
\item \emph{Architecture} is a \textbf{conceiving, planning, and designing structures}
\begin{itemize}
\item CA has purpose only when given SW
\end{itemize}
\item \emph{Software} is a \textbf{description of a computation} expressed in a programming language, any data, and documentation
\begin{itemize}
\item Purpose 1: Definining an DS \& A
\item Purpose 2: Executing
\end{itemize}
\item \emph{Interpreter} \textbf{executes software}
\begin{itemize}
\item Directly executes instructions expressed in a PL
\item \textbf{Does NOT rely on "Turtles all the way down"} (interpreter for interpreter for interpreter\ldots{}) approach
\end{itemize}
\item \emph{Compiling} is the process of \textbf{traslating} programs written in one \textbf{HLL} (High-level language) into a \textbf{LLL} that \textbf{has a machine interpreter}
\end{itemize}

\subsection{C Compiling Process}
\label{sec:org9030018}

\begin{verbatim}
source_code -> preprocessor -> preprocessed source code -> compiler -> assembly code -> assembler -> relocatable object code -> linker (w/ object code in lib) -> binary object code
\end{verbatim}

\subsection{Mechanical Computers}
\label{sec:org3c16917}

\begin{itemize}
\item Antikythera Mechanism (200B.C): Count Olumpics days
\item Charles Babbage (1849)
\end{itemize}

\subsubsection{Disadvantages}
\label{sec:orgf460e20}

\begin{itemize}
\item Parts are small, require individual assembly
\item Part shape and size determine computational function
\item Parts cause waer and accuracy degrades over time
\item Algorithm are slow
\end{itemize}

\subsection{Vacuum Tube Computers}
\label{sec:orgad836db}

\begin{itemize}
\item Colossus
\end{itemize}

\subsubsection{Disadvantages}
\label{sec:org2814d9c}

\begin{itemize}
\item About the same volume as mechanical computer
\item Uses a lot of electrical energy
\item Vacuum tubes burn out
\end{itemize}

\subsection{Transistor}
\label{sec:org0d3bf0d}

\begin{itemize}
\item First one built at AT\&T Bell Labs
\item Used to use germanium crystal, now use silicon
\item Futures are graphene or single layer of carbon
\end{itemize}

\subsection{Two Architectures}
\label{sec:org346516e}

\subsubsection{Harvard Architecture}
\label{sec:orgeebcadf}

\textbf{Separate memories} for instructions and data

\subsubsection{Von Neumann Architecture}
\label{sec:orgbc69deb}

\textbf{Single memory} for instruction and data

\section{Representation}
\label{sec:org2be4f8e}
\end{document}