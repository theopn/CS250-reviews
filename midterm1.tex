% Created 2022-09-19 Mon 00:52
% Intended LaTeX compiler: pdflatex
\documentclass[11pt]{article}
\usepackage[utf8]{inputenc}
\usepackage[T1]{fontenc}
\usepackage{graphicx}
\usepackage{longtable}
\usepackage{wrapfig}
\usepackage{rotating}
\usepackage[normalem]{ulem}
\usepackage{amsmath}
\usepackage{amssymb}
\usepackage{capt-of}
\usepackage{hyperref}
\author{Theo Park}
\date{\today}
\title{CS250 Midterm 1 Review}
\hypersetup{
 pdfauthor={Theo Park},
 pdftitle={CS250 Midterm 1 Review},
 pdfkeywords={},
 pdfsubject={},
 pdfcreator={Emacs 28.1 (Org mode 9.5.2)}, 
 pdflang={English}}
\begin{document}

\maketitle
\setcounter{tocdepth}{2}
\tableofcontents


\section{Why Computer Architecture}
\label{sec:orgc8a0220}

\subsection{Definitions}
\label{sec:org80b6dfb}

\begin{itemize}
\item \emph{Computer} is a machine that can be programmed to \textbf{carry out computation automatically}
\item \emph{Architecture} is a \textbf{conceiving, planning, and designing structures}
\begin{itemize}
\item CA has purpose only when given SW
\end{itemize}
\item \emph{Software} is a \textbf{description of a computation} expressed in a programming language, any data, and documentation
\begin{itemize}
\item Purpose 1: Definining an DS \& A
\item Purpose 2: Executing
\end{itemize}
\item \emph{Interpreter} \textbf{executes software}
\begin{itemize}
\item Directly executes instructions expressed in a PL
\item \textbf{Does NOT rely on "Turtles all the way down"} (interpreter for interpreter for interpreter\ldots{}) approach
\end{itemize}
\item \emph{Compiling} is the process of \textbf{traslating} programs written in one \textbf{HLL} (High-level language) into a \textbf{LLL} that \textbf{has a machine interpreter}
\end{itemize}

\subsection{C Compiling Process}
\label{sec:org08285fc}

// TODO Find a better way to put diagram in Org files
\begin{verbatim}
source_code -> preprocessor -> preprocessed source code -> compiler -> assembly code -> assembler -> relocatable object code -> linker (w/ object code in lib) -> binary object code
\end{verbatim}

\begin{itemize}
\item Preprocessed Source Code: Does not contain \textbf{comments, macros, includes}, etc
\item Assembly Code: \textbf{Machine specific}
\end{itemize}

\subsection{Mechanical Computers}
\label{sec:orgba3713a}

\begin{itemize}
\item Antikythera Mechanism (200B.C): Count Olumpics days
\item Charles Babbage (1849)
\end{itemize}

\subsubsection{Disadvantages}
\label{sec:orgc1e6b6a}

\begin{itemize}
\item Parts are small, require individual assembly
\item Part shape and size determine computational function
\item Parts cause waer and accuracy degrades over time
\item Algorithm are slow
\end{itemize}

\subsection{Vacuum Tube Computers}
\label{sec:org807699f}

\begin{itemize}
\item Colossus
\end{itemize}

\subsubsection{Disadvantages}
\label{sec:org07bfcb1}

\begin{itemize}
\item About the same volume as mechanical computer
\item Uses a lot of electrical energy
\item Vacuum tubes burn out
\end{itemize}

\subsection{Transistor}
\label{sec:org010ec42}

\begin{itemize}
\item First one built at AT\&T Bell Labs
\item Used to use germanium crystal, now use silicon
\item Futures are graphene or single layer of carbon
\end{itemize}

\subsection{Two Architectures}
\label{sec:orgc586793}

\subsubsection{Harvard Architecture}
\label{sec:org152fc90}

\textbf{Separate memories} for instructions and data

\subsubsection{Von Neumann Architecture}
\label{sec:org333747e}

\textbf{Single memory} for instruction and data

\section{Representation}
\label{sec:org6ad20ca}

\subsection{Electrical Representation of Bits}
\label{sec:org1df6ba4}

\begin{itemize}
\item \textbf{V (max) voltage V - \(\Delta\)} is recognizes as 1
\item \textbf{0 to 0 + \(\delta\)} is recognizes as 0
\item \textbf{Rising edge} and \textbf{falling edge} are ignored
\end{itemize}

\subsection{Bit String}
\label{sec:orge795f23}

\begin{itemize}
\item \textbf{Bus: Collection of k wires carrying k-bits}
\item \textbf{k-bits} on \textbf{k-wires}
\item k-bits can represent up to \textbf{\(2^k\) values}
\item \emph{Bit strings are only meaningful when it is paried with a representation}
\end{itemize}

\section{Regular Representations}
\label{sec:org8fe1875}

Unsigned and 2's complement integers are native data types for most modern circuits

\subsection{Unsigned integer, base 2, weighted positional}
\label{sec:org8d5d100}

\textbf{Regular binary number} that we think of normally.

\texttt{001011} = \(0 \times 2^5 + 0 \times 2^4 + 1 \times 2^3 + 0 \times 2^2 + 1 \times 2^1 + 1 \times 2^0 = 11\)

\subsection{Sign Magnitude}
\label{sec:orgb9c460a}

\textbf{UIB2WP but the MSB is the sign} (MSP = left most bit).

\texttt{101011} = \(-1(0 \times 2^4 + 1 \times 2^3 + 0 \times 2^2 + 1 \times 2^1 + 1 \times 2^0) = -11\)

\subsubsection{Characteristics of sign magnitude}
\label{sec:org6791e67}

\begin{itemize}
\item There are two zeros (0000 = +0, 1000 = -1)
\item Less number can be represented (duh)
\end{itemize}

\subsection{Two's Complement}
\label{sec:org4633eac}

\textbf{MSB weight is negative}

\texttt{101011} = \(-(1 \times 2^4) + 1 \times 2^3 + 0 \times 2^2 + 1 \times 2^1 + 1 \times 2^0) = -5\)

\subsubsection{Characteristics of two's complement}
\label{sec:org496655b}

\begin{itemize}
\item Only one bit string for zero
\item \textbf{Invert bit string and add 1 to get the negative}
\item \textbf{Uses the same circuit as unsigned integer add/subtraction}
\end{itemize}

\section{Casting/Sign Extension}
\label{sec:org0145952}

\begin{itemize}
\item Unsigned integer: \textbf{Add 0 in front}
\item 2's complement: \textbf{Add MSB in front}
\end{itemize}

\section{Overflow}
\label{sec:org3494566}

\begin{itemize}
\item Adding two k-bit unsigned integer resulting in (k+1)-bit result
\item \(A +_k B = (A+B) \text{ mod } k\) prevents it
\end{itemize}

\section{Gray Code}
\label{sec:orgc27a5d9}

For sensors where bits need to be detected fast, "gray code" where only one bit changes per number is used.

\section{ASCII}
\label{sec:org11908b4}

\subsection{History}
\label{sec:org87d64c8}

\emph{Baudot Code} in 1870 used to represent \(2^5\) characters with 5 keys.

\subsubsection{Design of ASCII}
\label{sec:org4db9cdf}

\begin{itemize}
\item \textbf{Designed for machine, not human}
\item Alphabetic order = integer order of chracter codes
\item Upper and lower case only differ in \textbf{bit 7, the MSB}
\end{itemize}

\subsubsection{Unicode}
\label{sec:orga224ea9}

\begin{itemize}
\item Up to 4 bytes per character
\item Currently 14.0, supports emoji
\end{itemize}

\section{Order of Bytes in Memory}
\label{sec:org6fcaf0e}

\begin{itemize}
\item \emph{Big Endian}: \textbf{MSB comes first}
\texttt{0x5060} is stored as \texttt{0x5060}
\item \emph{Lil Endian}: \textbf{LSB comes first}
\texttt{0x5060} is stored as \texttt{0x6050}
\end{itemize}

\section{Floating Point Representation (IEEE 754)}
\label{sec:orgd573ec0}

\texttt{|S| Exponent | mantissa |}

\subsection{Exponent}
\label{sec:orgccf78ee}

Exponent is a biased integer. The initial range is -127 < e < 127, sign is made implicit by E = e + Bias = e + 127.

\subsection{Mantissa}
\label{sec:org714f1df}

Unless the number is 0, the MSB of the mantissa must be 1 -> No need to store! (\textbf{hidden bit}).
Instead, \textbf{one extra precision bit} is stored in the end.

\subsection{Runtime Anomalies}
\label{sec:orgcd37fc0}

\begin{enumerate}
\item \emph{E = 0, Mantissa = 0,}: \(\pm 0\), depending on the sign bit
\item \emph{E = 0, Mantissa \(\neq\) 0, Mantissa MSB = 0}: De-normalized number, gradual underflow
\item \emph{E = 255, Mantissa = 0}: \(\pm \infty\); in general, overflow is set to infinity to help people
\item \emph{E = 255, Mantissa \(\neq\) 0}: Not a Number
\end{enumerate}
\end{document}