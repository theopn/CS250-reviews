% Created 2022-09-18 Sun 17:27
% Intended LaTeX compiler: pdflatex
\documentclass[11pt]{article}
\usepackage[utf8]{inputenc}
\usepackage[T1]{fontenc}
\usepackage{graphicx}
\usepackage{longtable}
\usepackage{wrapfig}
\usepackage{rotating}
\usepackage[normalem]{ulem}
\usepackage{amsmath}
\usepackage{amssymb}
\usepackage{capt-of}
\usepackage{hyperref}
\author{Theo Park}
\date{\today}
\title{CS250 Midterm 1 Review}
\hypersetup{
 pdfauthor={Theo Park},
 pdftitle={CS250 Midterm 1 Review},
 pdfkeywords={},
 pdfsubject={},
 pdfcreator={Emacs 28.1 (Org mode 9.5.2)}, 
 pdflang={English}}
\begin{document}

\maketitle
\setcounter{tocdepth}{2}
\tableofcontents


\section{Why Computer Architecture}
\label{sec:org700f66c}

\subsection{Definitions}
\label{sec:org936cb10}

\begin{itemize}
\item \emph{Computer} is a machine that can be programmed to \textbf{carry out computation automatically}
\item \emph{Architecture} is a \textbf{conceiving, planning, and designing structures}
\begin{itemize}
\item CA has purpose only when given SW
\end{itemize}
\item \emph{Software} is a \textbf{description of a computation} expressed in a programming language, any data, and documentation
\begin{itemize}
\item Purpose 1: Definining an DS \& A
\item Purpose 2: Executing
\end{itemize}
\item \emph{Interpreter} \textbf{executes software}
\begin{itemize}
\item Directly executes instructions expressed in a PL
\item \textbf{Does NOT rely on "Turtles all the way down"} (interpreter for interpreter for interpreter\ldots{}) approach
\end{itemize}
\item \emph{Compiling} is the process of \textbf{traslating} programs written in one \textbf{HLL} (High-level language) into a \textbf{LLL} that \textbf{has a machine interpreter}
\end{itemize}

\subsection{C Compiling Process}
\label{sec:org29ea6c1}

// TODO
\begin{verbatim}
source_code -> preprocessor -> preprocessed source code -> compiler -> assembly code -> assembler -> relocatable object code -> linker (w/ object code in lib) -> binary object code
\end{verbatim}

\begin{itemize}
\item Preprocessed Source Code: Does not contain \textbf{comments, macros, includes}, etc
\item Assembly Code: \textbf{Machine specific}
\end{itemize}

\subsection{Mechanical Computers}
\label{sec:org2d1f3b9}

\begin{itemize}
\item Antikythera Mechanism (200B.C): Count Olumpics days
\item Charles Babbage (1849)
\end{itemize}

\subsubsection{Disadvantages}
\label{sec:org05388cc}

\begin{itemize}
\item Parts are small, require individual assembly
\item Part shape and size determine computational function
\item Parts cause waer and accuracy degrades over time
\item Algorithm are slow
\end{itemize}

\subsection{Vacuum Tube Computers}
\label{sec:org248702e}

\begin{itemize}
\item Colossus
\end{itemize}

\subsubsection{Disadvantages}
\label{sec:org043bd06}

\begin{itemize}
\item About the same volume as mechanical computer
\item Uses a lot of electrical energy
\item Vacuum tubes burn out
\end{itemize}

\subsection{Transistor}
\label{sec:orgf9585b3}

\begin{itemize}
\item First one built at AT\&T Bell Labs
\item Used to use germanium crystal, now use silicon
\item Futures are graphene or single layer of carbon
\end{itemize}

\subsection{Two Architectures}
\label{sec:org4448219}

\subsubsection{Harvard Architecture}
\label{sec:org95d1259}

\textbf{Separate memories} for instructions and data

\subsubsection{Von Neumann Architecture}
\label{sec:org7aaa20a}

\textbf{Single memory} for instruction and data

\section{Representation}
\label{sec:org2c776e8}

\subsection{Electrical Representation of Bits}
\label{sec:org420444a}

\begin{itemize}
\item \textbf{V (max) voltage V - \(\Delta\)} is recognizes as 1
\item \textbf{0 to 0 + \(\delta\)} is recognizes as 0
\item \textbf{Rising edge} and \textbf{falling edge} are ignored
\end{itemize}

\subsection{Bit String}
\label{sec:org23b0f08}

\begin{itemize}
\item \textbf{Bus: Collection of k wires carrying k-bits}
\item \textbf{k-bits} on \textbf{k-wires}
\item k-bits can represent up to \textbf{\(2^k\) values}
\item \emph{Bit strings are only meaningful when it is paried with a representation}
\end{itemize}

\section{Regular Representations}
\label{sec:orgac5b3cc}

\subsection{Unsigned integer, base 2, weighted positional}
\label{sec:org2cec9a3}

\emph{Regular binary number} that we think of normally.

\texttt{001011} = \(0 \times 2^5 + 0 \times 2^4 + 1 \times 2^3 + 0 \times 2^2 + 1 \times 2^1 + 1 \times 2^0 = 11\)

\subsection{Sign Magnitude}
\label{sec:orgdb54637}

\emph{UIB2WP but the MSB is the sign}. MSP = left most bit.

\texttt{101011} = \(-1(0 \times 2^4 + 1 \times 2^3 + 0 \times 2^2 + 1 \times 2^1 + 1 \times 2^0) = -11\)

\subsubsection{Disadvantages of sign magnitude}
\label{sec:org08eafdd}

\begin{itemize}
\item There are two zeros (0000 = +0, 1000 = -1)
\item Less number can be represented (duh)
\end{itemize}

\subsection{Two's Complement}
\label{sec:org71029fa}
\end{document}